\chapter{Einleitung (chapter)}
Die aktuellen Word-Vorlage befindet sich im Relax und wird dort gepflegt: \url{https://relax.reutlingen-university.de/mod/resource/view.php?id=109791}. Änderung in Word sollten in der  Latex-Vorlage nachgezogen werden.

Es ist zu beachten, dass das Formatieren des Textes als einer der letzten
Schritte der Ausarbeitung durchgeführt wird, da dieser Schritt erfahrungsgemäß viel Zeit in Anspruch nimmt und daher nur einmalig ausgeführt werden sollte.
Der vorliegende Entwurf wurde mit \LaTeX ~erstellt.

\section{Gliederung des Textes (section)}
Nach jedem der nachfolgend genannten Abschnitte muss mit einer neuen
Seite begonnen werden. Die Gestaltung der in diesem Dokument
enthaltenen Abschnitte gilt als Vorlage für die Seminararbeit \cite{BlindBuch}.
\begin{itemize}
\item Titelseite
\item Inhaltsverzeichnis
\item Abbildungsverzeichnis
\item Abkürzungsverzeichnis
\item Glossar
\item Textkörper (Ausarbeitung)
\item Literaturverzeichnis
\item Eidesstattliche Erklärung
\end{itemize}
Die Abschnitte der Verzeichnisse können in einer Seite aufeinanderfolgend dargestellt werden. Der Textkörper (Abschnitte, welche den Verzeichnissen folgen) beginnt grundsätzlich auf einer neuen Seite.

\begin{table}
\begin{tabular}{ l c r }
  1 & 2 & 3 \\
  4 & 5 & 6 \\
  7 & 8 & 9 \\
\end{tabular}
\caption{Das ist eine einfach Tabelle}
\label{tab:tabelle1}
\end{table}
 
\subsection{Abbildungen (subsection)}
 \lipsum[1]
 
\begin{figure}[H]
	\centering
	\includegraphics[width=1\textwidth]{Bilder/BlindBilder/Notavailable.jpg}
	\caption{Ich bin ein Blindbild und verweise auf mich selbst \ref{Blinblid}}
	\label{Blinblid}
\end{figure}