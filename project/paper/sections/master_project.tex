\label{chap:master_project}
Neben der Entwicklung der Anwendung zur \acl{ME}, standen einige Termine für das IoT-Lab an. Unter anderem war das Thema auf der WVK.15\footnote{\url{http://wvk.reutlingen-university.de/index.php?site=topic&id=150299}}, der InformaticsInside \footnote{\url{http://www.infoinside.reutlingen-university.de/index.php?site=program}}, am Tag der offenen Tür und dem Studientag präsent. 
Die in diesem Dokument genutzte \LaTeX -Vorlage für die Ausarbeitung\footnote{\url{https://relax.reutlingen-university.de/course/view.php?id=6884}}, wurde an den Stand der vorgegebenen Word-Vorlage angepasst und zur Verfügung gestellt.

Die Einarbeitung und vor allem die Dokumentation des Fahrsimulators war eine weitere wichtige Aufgabe. Viele Fragen waren nach dem Weggang von Emre Yay nicht abschließend geklärt. Es war also viel Suchen, Debuggen und Nachfragen notwendig. Um anderen diesen Aufwand zu ersparen war es wichtig den Simulator zu dokumentieren. Dazu musste zuerst das Format entschieden werden, da mehrere Personen, am besten auch parallel, an der Dokumentation arbeiten sollten. Ein Wiki\footnote{\url{http://iotlab.reutlingen-university.de/iotlab_wiki/index.php/Driving_Simulator}} in der Umgebung der der IoT-Website erschien am besten für diese Aufgabe. Die Struktur und erster Inhalt waren die ersten Aufgaben für die Dokumentation. Weiterhin sind erste Schritte dokumentiert, sodass ein Neuling einen Einstiegspunkt für einfache Aufgaben (bspw. Simulation starten) hat. Juniors im Masterprojekt konnten bereits von einer Einführung profitieren.

Während der Entwicklung gab es immer wieder Kommunikationsbedarf mit verschiedenen Supportstellen. So traten Probleme mit OpenDS, CANoe und Emokit auf, die mit dem jeweiligen Support per Email oder Telefon geklärt werden mussten. Den meisten Aufwand erzeugt in dieser Hinsicht das einschicken des EEG Headsets. Nachdem der Defekt eines Sensors Anfang Februar festgestellt wurde, war nach einigen Emails mit dem Emotiv Support und in Absprache mit MKI Service klar, dass das Headset eingeschickt werden muss. Dies passiert dann Anfang März und verzögerte sich nochmal wegen Problemen am Zoll. Ende April kam das Headeset dann im TechCenter der Firma an und mehrere Nachfragen lieferten keine neuen Erkenntnisse. Ende Mai kam die Nachricht, dass der Fehler gefunden und behoben wäre. Da dieser Fall jedoch vorher getestet wurde, wurde dies Emotiv mitgeteilt mit der Bitte noch einmal zu testen. Beim zweiten Test wurde dann festgestellt, dass das Headset nicht stabil läuft und es wohl ein anderes Problem war. Nach weiteren 3 Wochen entschied man sich, ein neues Gerät zu verschicken. Weitere Nachfragen brachten dann auch die Herausgabe eines Tracking-Codes und das EEG ist auf dem Weg. Die Notwendigkeit zu häufigen Nachfragen und die Reaktionszeiten von jeweils mindestens einer Woche, waren frustrierend und zeitraubend.
