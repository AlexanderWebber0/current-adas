\label{sec:extraction}
Nach der Aufbereitung und Verarbeitung der EEG-Signale, müssen nun signifikante Daten extrahiert und eine Merkmalsmenge erstellt werden, welche vom Klassifikator erkannt werden kann. 

Aus der validierten Hypothese aus dem Experiment (vgl. Kapitel \ref{chap:data}) folgt, dass es eine Veränderung vom Anfang zum Ende der Testfahrt geben muss. Um den Effekt zu verstärken, wurden Datensätze aus den Minuten 5-10 (Wach) und 20-25 (Müde) entnommen und jeweils verglichen. Es wurde eine fließende Veränderung angenommen, welche unter Umständen schwerere Erkennbar wäre. Die beiden Datensätze wurden durch die Verarbeitungskette geschickt und in einer Testmenge gespeichert. 

Wie im vorherigen Kapitel gezeigt, lassen Veränderungen in den Frequenzbändern verschiedene Schlüsse zu. So konnten Pal et al. \cite{Pal2008} Zusammenhänge zwischen Veränderungen der Alpha- und Theta-Wellen und Veränderungen der kognitiven Fähigkeiten feststellen. Dies wurde bei der Analyse der Daten berücksichtigt und besonders auf eine Veränderung der Theta- und Alpha-Wellen geachtet. 

Mehrere Merkmale wurden Testweise extrahiert und manuell auf Unterschiede geprüft. Werte aus der Spracherkennung wie Nulldurchgangsrate und Signalenergie wurden für das verarbeitete Signal, als auch für die Frequenzbänder durchgeführt. Die Ergebnisse lieferten keine eindeutigen Unterschiede. Statistische Merkmale, wie die Standardabweichung und die Varianz waren ebenfalls nicht eindeutig. 

Kleinere Abweichungen in der Ausschlagshöhe der Frequenzbänder konnten über die gesamten Datensätze Wach und Müde erkannt werden. Am deutlichsten waren die Veränderungen an den Sensoren "`AF3"', "`AF4"', "`F3"', "`F4"', "`F7"', "`F8"' und dort in den Theta-Wellen erkennbar. Um mögliche Fehlerquellen zu verringern und die Erkennung zu vereinfachen, wurden Sequenzen mit größeren Ausreißern manuell bereinigt. Da sich die Unterschiede in der Ausschlagshöhe nur auf die gesamte Zeitdauer von 5min erstreckten und nicht auf ein Zeitfenster von 1s, war zu prüfen, ob der Klassifikator in der Lage ist, die beide Zustände sicher zu erkennen.