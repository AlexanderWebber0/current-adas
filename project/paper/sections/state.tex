\label{chap:state}
Systeme zur Erkennung von Müdigkeit versuchen mit verschiedenen Parametern herauszufinden, ob sich die Person (der Fahrer) noch in einem aufmerksamen Zustand befindet. Die lässt sich in drei Bereiche unterteilen: Fahrverhalten, Computer-Vision (CV) und Körpersensoren. 
\comp

Ansätze mit einem EKG allein, zeigten in verschiedenen Arbeiten kein eindeutiges Ergebnis \cite{Vicente_6164509}\cite{Rogado_4913155}. Beide versuchten Informationen aus der Herzfrequenzvariabilität zu erhalten und diese zu klassifizieren. 
Khushaba et al. \cite{Khushaba_5580017} versuchten EEG, EKG und EOG zu verbinden und verglichen verschiedenen Kombinationen. Mit einem "`fuzzy wavlet"' basierten Algorithmus wurde die Signale aufbereitet und zeigten, dass ein EEG alleine bereits ausreicht. Die Kombination eines EEG mit EKG bzw. EOG verbesserte das Ergebnis nicht signifikant. Auch Johnson et. al \cite{Johnson11} kamen zu der Erkenntnis, dass ein EEG ausreicht und das eingesetzte EOG nicht benötigt wird. Subasi \cite{Subasi:2005:ARA:1707423.1707550} konnte mit einem EEG die Zustände "`Wach"', "`Schläfrig"' und "`Schlafend"' unterscheiden. Die Wavelet-Transformation und das genutzte künstliche Neuronale Netz (KNN) führten zu einer Erkennungsrate von 93\%. Vuckovic et al. \cite{Vuckovic2002349} fanden den besten Algorithmus für die Initialisierung des KNNs: Der Learning Vector Quantization Algorithmus. Im Vergleich mit EEG-Experten erreichte das KNN eine Übereinstimmung von 90\%. Huang et al. \cite{Huang_548971} nutzten ein Hidden Markov Modell zu Erkennung und erreichten eine gute Erfolgsrate. Lin et al. \cite{Lin05eeg-baseddrowsiness} nutzten die Unabhängige Komponenten Analyse (UKA) und Lineare Regression (LR) und konnten zeigen, dass hiermit  bis zu 88\% richtige Ergebnisse erzielt werden können. 


Die betrachteten Arbeiten unterstreichen die Eignung von Körpersensoren  um Müdigkeit zu erkennen. Die Stärken im Bereich Präzision und Korrektheit, im Vergleich zu Verhaltensanalyse oder CV-Techniken, zeigten sich in der Erkennungsrate. Den Komfort betreffend bleiben Körpersensoren jedoch hinter den anderen Ansätzen zurück. 

Verhaltensanalyse oder CV-Technik werden für die Arbeit nicht berücksichtigt und treten nur bei der Analyse / Markierung der Testdaten in Erscheinung. Gähnen, häufiges blinzeln oder abkommen von der Fahrspur, können Hinweise auf Müdigkeit sein, sodass die entsprechenden EEG-Sequenzen gelabelt werden können. 
Aufgrund seiner höheren Genauigkeit, gegenüber EKG oder EOG, wird für die Anwendung ein EEG genutzt. Ein multimodales System ist vorbereitet, aber nicht umgesetzt. Die betrachteten Arbeiten zeigen in verschiedensten Ausführungen sehr gute Ergebnisse (\textasciitilde 90\% \cite{Lin05eeg-baseddrowsiness}, \cite{Subasi:2005:ARA:1707423.1707550}), sodass es sich beim EEG um eine aussichtsreiche Grundlage handelt. 
Bei der Klassifizierung lässt nutzen viele Ansätze ein Künstliches Neuronales Netz  (KNN)\cite{Subasi:2005:ARA:1707423.1707550}\cite{Vuckovic2002349}\cite{wilson_890161}\cite{khalifa_893852}. Weitere Ansätze nutzen die Lineare Diskriminanten Analyse\cite{Vicente_6164509}\cite{Khushaba_5580017} oder ein SVM\cite{Park:2009:DDD:1667780.1667798}\cite{zhang_6513058}. Aufgrund der Häufigkeit in den anderen Arbeiten und der guten Bibliotheksunterstützung (PyBrain) wurde der Ansatz mit einem KNN weiter verfolgt. Die Vorgestellten Lösungen arbeiten mit medizinischen oder selbst gebauten EEGs. Alle Ansätze wurden in einer simulierten Umgebung entwickelt und nie unter realen Bedingungen getestet. Aus der Analyse der verwandten Forschungsarbeiten ergeben sich Anforderungen an eine Anwendung zur \acl{ME}. 