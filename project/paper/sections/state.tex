\label{chap:state}
Systeme zur Erkennung von Müdigkeit versuchen mit verschiedenen Parametern herauszufinden, ob sich die Person (der Fahrer) noch in einem aufmerksamen Zustand befindet. Die lässt sich in drei Bereiche unterteilen: Fahrverhalten, Computer-Vision (CV) und Körpersensoren. 

In der Praxis setzen Automobilhersteller wie Daimler\cite{Daimler} und Volkswagen, sowie Automobilzulieferer wie Bosch\cite{Bosch} auf die Analyse des Fahrverhaltens. Insbesondere Spurhalten und ruckartiges Gegenlenken scheinen ein signifikantes Indiz für beginnende Übermüdung zu sein. Weiterhin sind externe Geräte und einige Apps für Smartphones erhältlich. Leider existieren kaum öffentlich zugängliche Arbeiten zu diesem Ansatz von  \acl{ME}, da es sich um interne Entwicklungen handelt. 

Beim CV-Ansatz wird der Fahrer und die Straße mit Hilfe von Kameras beobachtet. Zhang et al. \cite{Zhang:2015:RSD:2753829.2629482} stellen hierzu eine Applikation mit der Microsoft Kinect vor. Es wurden sowohl die Kopfpose, als auch die Augenstatus bestimmt. Bergasa et al. \cite{Bergasa_1603553} extrahierten aus dem Bild einer Infrarot Kamera mehrere Features, wie \acl{bspw} den prozentualen Anteil von geschlossenen Augen (Percent eye closure, PERCLOS). Mit dieser Technik erreichten sie bei der Erkennung von Übermüdung eine nahezu hundertprozentige Erfolgsrate. Kamerabasierte Systeme schränkenden Fahrer nicht ein, da kein direkter Kontakt zum Fahrer bestehen muss. Jedoch ist jede Kamera optischen Grenzen unterworfen (bspw. bei Nacht oder schlechtem Wetter).

Die beiden beschriebenen Bereich werden in dieser Arbeit nur für die manuelle Markierung der aufgenommen Datensätzen genutzt. Für die Erkennung von Müdigkeit werden verschiedene Körpersensoren bzw. deren Kombination (multimodal) eingesetzt. Meist werden elektrische Spannung am und im Körper gemessen. Neben dem EEG, werden bspw. die Elektrokardiographie (EKG) oder Elektrookulografie (EOG) genutzt. Beim EKG wird die elektrische Aktivität des Herzmuskels erfasst, um bspw. die  Herzfrequenz zu bestimmen. Das EOG misst die Bewegung der Augen, um bspw. Blinzeln zu erkennen.

Ansätze mit einem EKG allein, zeigten in verschiedenen Arbeiten kein eindeutiges Ergebnis \cite{Vicente_6164509}, \cite{Rogado_4913155}. Beide versuchten Informationen aus der Herzfrequenzvariabilität zu erhalten und diese zu klassifizieren. 
Khushaba et al. \cite{Khushaba_5580017} versuchten EEG, EKG und EOG zu verbinden und verglichen verschiedenen Kombinationen. Mit einem "`fuzzy wavlet"' basierten Algorithmus wurde die Signale aufbereitet und zeigten, dass ein EEG alleine bereits ausreicht. Die Kombination eines EEG mit EKG bzw. EOG verbesserte das Ergebnis nicht signifikant. Auch Johnson et. al \cite{Johnson11} kamen zu der Erkenntnis, dass ein EEG ausreicht und das genutzte EOG nicht benötigt wird. 
Subasi \cite{Subasi:2005:ARA:1707423.1707550} konnte mit einem EEG die Zustände "`Wach"', "`Schläfrig"' und "`Schlafend"' unterscheiden. Die Wavelet-Transformation und das genutzte künstliche Neuronale Netz (KNN) führten zu einer Erkennungsrate von 93\%. Vuckovic et al. fanden den besten Algorithmus für die Initialisierung des KNNs: Der Learning Vector Quantization Algorithmus. Im Vergleich mit EEG-Experten erreichte das KNN eine Übereinstimmung von 90\%. Huang et al. \cite{Huang_548971} nutzten ein Hidden Markov Modell zu Erkennung und erreichten eine gute Erfolgsrate. 
Lin et al. \cite{Lin05eeg-baseddrowsiness} nutzten die Unabhängige Komponenten Analyse (UKA) und Lineare Regression (LR) und konnten zeigen, dass hiermit  bis zu 88\% richtige Ergebnisse erzielt werden können. 

Die betrachteten Arbeiten unterstreichen die Eignung von Körpersensoren  um Müdigkeit zu erkennen. Die Stärken im Bereich Präzision und Richtigkeit der Ergebnisse, im Vergleich zu Verhaltensanalyse oder CV-Techniken, zeigte sich in der Analyse. In Sachen Komfort bleiben  Körpersensoren jedoch hinter den anderen Ansätzen zurück.
Das EEG zeigte im Vergleich zu anderen Körpersensoren bessere Ergebnisse und somit ein geeignetes Vorgehen um Müdigkeit im Fahrzeugumfeld zu erkennen. Die Vorgestellten Lösungen arbeiten mit medizinischen oder selbst gebauten EEGs. Diese sind eher unhandlich und eigenen sich nicht für den Einsatz in einem echten Fahrzeug (Solche Tests wurden nicht durchgeführt).