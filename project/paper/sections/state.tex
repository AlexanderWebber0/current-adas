\label{chap:state}
Systeme zur Erkennung von Müdigkeit versuchen mit verschiedenen Parametern herauszufinden, ob sich die Person (der Fahrer) noch in einem aufmerksamen Zustand befindet. Die lässt sich in drei Bereiche unterteilen: Fahrverhalten, Computer-Vision (CV) und Körpersensoren. Die beiden ersten Bereich werden nur für die manuelle Markierung von Datensätzen genutzt. Die folgenden Ansätze beziehen sich ausschließlich auf die Erkennung mit Körpersensoren.

Für die Erkennung von Müdigkeit werden verschiedene Körpersensoren bzw. deren Kombination (multimodal) eingesetzt. Meist werden elektrische Spannung am und im Körper gemessen. 
Neben dem EEG, werden bspw. die Elektrokardiographie (EKG) oder Elektrookulografie (EOG) genutzt. Beim EKG wird die elektrische Aktivität des Herzmuskels erfasst, um bspw. die  Herzfrequenz zu bestimmen. Das EOG misst die Bewegung der Augen, um bspw. Blinzeln zu erkennen.

Ansätze mit einem EKG allein, zeigten in verschiedenen Arbeiten kein eindeutiges Ergebnis \cite{Vicente_6164509}, \cite{Rogado_4913155}. Beide versuchten Informationen aus der Herzfrequenzvariabilität zu erhalten und diese zu klassifizieren.

Khushaba et al. \cite{Khushaba_5580017} versuchten EEG, EKG und EOG zu verbinden und verglichen verschiedenen Kombinationen. Mit einem "`fuzzy wavlet"' basierten Algorithmus wurde die Signale aufbereitet und zeigten, dass ein EEG alleine bereits ausreicht. Die Kombination eines EEG mit EKG bzw. EOG verbesserte das Ergebnis nicht signifikant. Auch Johnson et. al \cite{Johnson11} kamen zu der Erkenntnis, dass ein EEG ausreicht und das genutzte EOG nicht benötigt wird. 

\cite{Subasi:2005:ARA:1707423.1707550} konnten mit einem EEG die Zustände "`Wach"', "`Schläfrig"' und "`Schlafend"' unterscheiden. Die Wavelet-Transformation und das genutzte künstliche Neuronale Netz (KNN) führten zu einer Erkennungsrate von 93\%. Vuckovic et al. fanden den besten Algorithmus für die Initialisierung des KNNs: Der Learning Vector Quantization Algorithmus. Im Vergleich mit EEG-Experten erreichte das KNN eine Übereinstimmung von 90\%. Huang et al. \cite{Huang_548971} nutzten ein Hidden Markov Modell zu Erkennung und erreichten eine gute Erfolgsrate.
Lin et al. \cite{Lin05eeg-baseddrowsiness} nutzten die Unabhängige Komponenten Analyse (UKA) und Lineare Regression (LR) und konnten zeigen, dass hiermit  bis zu 88\% richtige Ergebnisse erzielt werden können. 

Die betrachteten Arbeiten unterstreichen die Eignung des EEGs um Müdigkeit zu erkennen. Kombiniert mit anderen Sensoren können die Ergebnisse nur noch leicht verbessert werden.