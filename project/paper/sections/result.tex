\label{chap:result}
\textbf{TODO}
Da das EEG in den letzte 4 Monaten in Reparatur und somit nicht verfügbar war, sind die Ergebnisse bisher nicht aussagekräftig. Weder das Experiment, noch die Merkmalsextraktion und die Klassifizierung konnten sinnvoll evaluiert werden. 

Teil eins (1-4) der Anforderungen (vgl. Tab \ref{tab:requirements}) kann erst nach erfolgreicher Implementierung getestet werden. Die Portierung ist theoretisch sehr einfach möglich, wenn während der Fahrt ein Laptop genutzt wird (5). Die Handhabung und Komfort des Headsets ist im Vergleich zu medizinischen EEGs deutlich verbessert (6). Wie schon angenommen, ist es jedoch für den Produktiveinsatz ungeeignet. Das EEG lässt sich ähnlich wie eine Mütze tragen und ist sehr leicht, sodass man schnell vergisst, dass es da ist. Die kabellose Übertragung sorgt für maximale Beweglichkeit.

Tests auf einem Smartphone oder Tablet müssen gesondert erfolgen, da es unter anderem von den Treibern des EEG Headsets abhängt (7). Auch Lasttests sind erst nach fertiger Implementierung wirklich aussagekräftig (8). Der Einbau in die Simulationsumgebung ist bis zum eintragen der Werte im CAN-Bus umgesetzt. Das Herausnehmen der Werte ist noch nicht vollständig umgesetzt, da die Integration der CarInterface Anwendung nicht funktionierte (9). Die Anwendung lässt sich sehr gut auf verteilten Systemen ausführen, auch über den Anwendungsfall des Fahrsimulators hinaus. Die akquirierten Daten können via http an die Verarbeitungsschicht übertragen werden. Auch das Verschicken des Ergebnisses der Klassifizierung per http wäre denkbar und einfach umzusetzen (10). Die Art der Benachrichtigung über eine erkannte Müdigkeit ist derzeit noch nicht vollständig umgesetzt. Dem Fahrer wird entweder ein grüner oder roter Bildschirm angezeigt. Ob dies die passende Form ist, bleibt zu klären.

Die komplette Anwendung ist mit Docstring\footnote{\url{https://www.python.org/dev/peps/pep-0257/\#what-is-a-docstring}} versehen, aus denen eine html-Dokumentation erzeugt werden kann. Weiterhin sind schwierige Code-Teile mit einfachen Beispielen erweitert, um den Einstieg zu erleichtern. Ob es so möglich ist, als Neuling, selbständig die Anwendung zu verstehen bzw. zu erweitern hängt wohl auch von der jeweiligen Person ab. Werden jedoch Veränderungen vorgenommen, kann durch die Unit-Tests sichergestellt werden, dass die Klassen weiterhin das Richtige tun. Integrationstests wären eine sinnvolle  Erweiterung.