\label{chap:result}
In der vorgestellten Arbeit wurde eine Anwendung zur Müdigkeitserkennung entwickelt. Sie liest Rohdaten des EEG-Headsets, verarbeitet und klassifiziert sie. Die Datenerhebung ist lose gekoppelt und kann die EEG-Daten über mehrere Wege übertragen. Für die Verarbeitung der Daten stehen mehrere Klassen zur Verfügung, die zu einer Verarbeitungskette verbunden werden. Für die Klassifizierung wurde ein KNN eingesetzt, welches performant die EEG-Sequenzen einteilt. Die Anwendung fügt sich in die bestehende Simulationsumgebung der Reutlingen University ein.

Die Ergebnisse werden nun im einzelnen mit den Anforderungen (vgl. Tab \ref{tab:requirements}) verglichen.
Präzision und Genauigkeit (1) konnten nicht wie Erwartet umgesetzt werden. Die Erkennungsrate liegt derzeit bei ca. 66\%, das ist für ein sicherheitsrelevantes System deutlich zu niedrig. Es werden lediglich zwei von drei Sequenzen aus den besten Daten richtig erkannt. Die Gründe lassen sich nicht eindeutig klären. Es könnte bei der Datenaquise, dem Experiment, beginnen, an Fehlern oder falscher Berechnungen während der Verarbeitung oder ungünstigen Startparametern beim Training des KNN liegen. Da das Experiment bei Betrachtung objektiver Merkmale, sowie subjektiver Einschätzung der  Testpersonen selbst, den gewünschten Effekt hatte, wäre ein Fehler beim Headset und den gelieferten Daten zu suchen. Für die Validierung der Verarbeitungskette könnten etablierte EEG-Bibliotheken verwendet werden.
Fehlertoleranz (2) und Fehlerbehandlung (3) wurden umgesetzt und durch Unit- und Integrationstests abgesichert. 

Die Verarbeitungsgeschwindigkeit (4) der empfangen Daten beträgt  während eines Benchmarktests im Schnitt 200ms pro Sequenz und liegt damit deutlich unter der Liefergeschwindigkeit des EEGs (Sequenz $\equiv$ 128 Werte $\equiv$ 1000ms). 
Die Portierung des Systems ist theoretisch sehr einfach möglich, wenn während der Fahrt ein Laptop genutzt wird (5). Dies wurde aus Zeitgründen nicht durchgeführt. Die Handhabung und Komfort des Headsets ist im Vergleich zu medizinischen EEGs deutlich verbessert (6). Das EEG lässt sich ähnlich wie eine Mütze tragen, jedoch ist die Einrichtung der Sensoren aufwändiger als Erwartet (Gute Signal-Qualität für alle Sensoren). Die kabellose Übertragung sorgt für maximale Beweglichkeit. Wie schon angenommen, ist es jedoch für den Produktiveinsatz ungeeignet.

Tests auf einem Smartphone oder Tablet müssen gesondert erfolgen, da es unter anderem von den Treibern des EEG Headsets abhängt (7). Lasttests wurden nur auf dem Entwickler Laptop durchgeführt (8). Der Einbau in die Simulationsumgebung ist bis zum eintragen der Werte im CAN-Bus umgesetzt. Das Herausnehmen der Werte ist noch nicht vollständig umgesetzt, da bisher die Integration der CarInterface Anwendung nicht funktionierte (9). Die PoSDBoS Anwendung lässt sich sehr gut auf verteilten Systemen ausführen, auch über den Anwendungsfall des Fahrsimulators hinaus. Die akquirierten Daten können via http an die Verarbeitungsschicht übertragen werden. Auch das Verschicken des Ergebnisses der Klassifizierung per http wäre denkbar und einfach umzusetzen (10). Die Art der Benachrichtigung über eine erkannte Müdigkeit ist derzeit noch nicht vollständig umgesetzt. Dem Fahrer wird entweder ein grüner oder roter Bildschirm angezeigt. Ob dies die passende Form ist, bleibt zu klären.

Die komplette Anwendung ist mit Docstring\footnote{\url{https://www.python.org/dev/peps/pep-0257/\#what-is-a-docstring}} versehen, aus denen eine html-Dokumentation erzeugt werden kann. Weiterhin sind schwierige Code-Teile mit einfachen Beispielen erweitert, um den Einstieg zu erleichtern. Ob es so möglich ist, als Neuling, selbständig die Anwendung zu verstehen bzw. zu erweitern hängt wohl auch von der jeweiligen Person ab. Werden jedoch Veränderungen vorgenommen, kann durch die Unit-Tests sichergestellt werden, dass die Klassen weiterhin das Richtige tun. Integrationstests wären eine sinnvolle  Erweiterung.