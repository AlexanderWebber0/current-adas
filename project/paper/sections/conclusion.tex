\label{chap:conclusion}
Die Ergebnisse zeigen eine vielversprechende Grundlage und bilden das Grundgerüst für weitere Arbeiten. Die passende Architektur und Infrastruktur ist vorbereitet. Die Annahmen der Experiment Hypthese wurde bestätigt, wenngleich sie sich nicht im erwarteten Maße in den EEG Daten niedergeschlagen hatte.

Die unzureichende Genauigkeit ist der wichtigste Anhaltspunkt der weiteren Forschung. Es werden derzeit lediglich zwei von drei Sequenzen aus den besten Daten richtig erkannt. 
Die Gründe hierfür lassen sich nicht eindeutig klären. Es könnte bei der Datenaquise - dem Experiment - beginnen, an Fehlern oder falscher Berechnungen während der Verarbeitung oder ungünstigen Startparametern beim Training des KNN liegen. Das Experiment hatte, bei Betrachtung objektiver Merkmale, sowie subjektiver Einschätzung der Testpersonen selbst, den gewünschten Effekt. Für weitere Schritte könnten die Experimente wiederholt werden und ggf. über einen längeren Zeitraum (\textasciitilde 1h) laufen. So wäre der Unterschied zwischen Wach und Müde unter Umständen deutlicher erkennbar. Auch die Aufnahme von Testdaten in verschiedenen Szenarien (Abwechslungsreiche Tagfahrt und langweilige Nachtfahrt) und unterschiedlichen Startparametern (bspw. Schlafmenge vor dem Experiment), könnten die Unterschiede deutlicher machen.
Um Fehler beim Headset und den gelieferten Daten auszuschließen, 
müssen zuvor allerdings die Daten des Headsets und der genutzten Emokit Bibliothek noch einmal validiert werden.
Die Berechnungen Verarbeitungskette könnten mit etablierte EEG-Bibliotheken verglichen und werden. 