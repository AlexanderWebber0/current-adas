\ifincludespickzettel

\section{\LaTeX\ Spickzettel}
\label{sec:spickzettel}
\subsection{Mehr über \LaTeX\ erfahren}

Nachfolgend ein paar Quellen, um Dinge nachzuschlagen, die im Spickzettel nicht enthalten sind.

\begin{itemize}
	\item \url{http://en.wikibooks.org/wiki/LaTeX}
	\item \url{http://wiki.selfhtml.org/wiki/Kurse:LaTeX}	
	\item \url{http://tobi.oetiker.ch/lshort/lshort.pdf}
	\item \url{http://latex-project.org/guides/}
	\item \url{https://www.google.de/search?q=latex+something}
\end{itemize}

\subsection{Grundlegende Formatierungen}

\texttt{Schreibmaschine, Code}\\
\textbf{fetter Text}

In der Regel trennt Latex Wörter automatisch am Zeilenende, bei der Donau\-dampf\-schiff\-fahrtskapitäns\-mütze kann aber aber auch mal daneben gehen. Zur Optimierung kann man (sollte man sich aber für den Schluss aufheben) auch explizit trennen. Latex meckert überlange Zeilen an.

Abkürzungen wie \ac{ITIL} oder \acs{SQL} können zentral abgelegt, werden automatisch richtig eingesetzt und landen im Glossar.

Während dem Arbeiten kann es nützlich sein, sich \todo{noch genauer zu hinterfragende Aussagen} zu kennzeichnen. 
\reconsider{Selbes gilt für Abschnitte deren reine Existenz fragwürdig ist.}
Ein \textbackslash rem Kommentar-Block kann für Notizen genutzt werden.

\subsection{Zitate und Verweise}
So ist es. \cite{hug_algoverbessern}

\begin{quote}
Eine Menge ist die Zusammenfassung bestimmter, wohlunterschiedener
Objekte unserer Anschauung oder unseres Denkens, wobei von jedem
dieser Objekte eindeutig feststeht, ob es zur Menge gehört oder nicht. Die
Objekte der Menge heißen Elemente der Menge.
(\cite[Seite 36]{meinel_mathematische_2009})
\end{quote}

Es wird immer nur auf "`label"' verwiesen. Da \LaTeX Bilder, Tabellen und Listings\footnote{Listings sind bspw. Pseudocode} so platziert, dass die Seite möglichst voll ist, sollte man immer eine Referenz zum Element benennen (Siehe..., Klammern, wie in...).

Auf Bild verweisen: \autoref{img:logo}

Auf eine Tabelle verweisen (\autoref{tbl:tabelle})

Auf Code verweisen: \autoref{lst:registermaschine3}

Auf Kapitel verweisen. Siehe \autoref{sec:spickzettel}

\subsection{Formeln}

$a=\sqrt{9}$ (Formeln)

\begin{equation}
  1=4-3
\end{equation}

und Reihen

\begin{eqnarray}
  2=7-5\\
  3=2-1
\end{eqnarray}

\subsection{Bilder}
\begin{figure}[h!]
  \begin{center}
    \includegraphics[width=2cm]{gilogo}
    \caption[Das Logo der GI]{\label{img:logo}Das Logo der GI \cite[Seite 36]{meinel_mathematische_2009}}
  \end{center}
\end{figure}

\subsection{Listen}

\begin{itemize}
	\item Liste 1
	\item Liste 2
\end{itemize}

\subsection{Tabellen}

\begin{table}[h!]
  \begin{center}
    \begin{tabular}{ | r | l }
      Heading 0.0  & Heading 0.1\\\hline
      Zelle 1.0  & Zelle 1.1\\
      Zelle 1.0  & Zelle 1.1\\
      Zelle 1.0  & Zelle 1.1\\
      \hline
	\end{tabular}
    \caption[Tabellenbeispiel]{\label{tbl:tabelle}Tabellenbeispiel \cite[Seite 36]{meinel_mathematische_2009}}
  \end{center}
\end{table}

\begin{lstlisting}[
	caption={[Das Potenzierungsprogramm]Das Potenzierungsprogramm \protect{\cite[Seite 39]{meinel_mathematische_2009}}},
	label=lst:registermaschine3,language=Python
]{}
def potenzierung(n, m):
   	z = 1
    while m != 0:
   	    z *= n
        	m -= 1
    return z
potenzierung(3, 2)
\end{lstlisting}

\subsection{Einen Bereich auskommentieren}
Perfekt für Randnotizen und Todos.

\rem{
Das kann man nicht sehen. 
}

Das wiederum kann man sehen.


\fi