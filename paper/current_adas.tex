\documentclass[8pt,a5paper]{acm_proc_article-sp}
% fix umlauts
\usepackage[ngerman]{babel}
\usepackage[utf8]{inputenc} 
\usepackage[T1]{fontenc}  % Times new Roman
\usepackage{mathptmx}
\usepackage[]{blindtext}
\usepackage[a5paper, left=1.70cm, right=1.20cm, top=1.30cm, bottom=1.40cm]{geometry}
\usepackage{epstopdf}
% balance columns. original from acm doesn't work
\usepackage{balance}
% colors
\usepackage[compact]{titlesec}
\usepackage{natbib}
\titlespacing{\section}{0pt}{*0.7}{*0.5}

\usepackage[nolist]{acronym}
 
%\usepackage[usenames,dvipsnames]{xcolor}
\usepackage[greyscale]{xcolor} 
% automatic crosslinks
\usepackage[hyphens]{url}
\usepackage[colorlinks=false,
            allbordercolors={0 0 0},
            pdfborderstyle={/S/U/W 0.5}]{hyperref}
%glossaries
%\usepackage{makeidx}
%\makeindex
%\usepackage[nomain]{glossaries}
%\makeglossaries
%\makeindex

\newcommand{\glspl}[1]{{#1}}

% http://en.wikibooks.org/wiki/LaTeX/Glossary
% http://mirror.informatik.uni-mannheim.de/pub/mirrors/tex-archive/macros/latex/contrib/glossaries/glossariesbegin.pdf

% Definitionsliste
\newcommand{\defitem}[1]{\item[#1]\phantomsection\label{#1}\hfill\\} 
\newcommand{\defref}[1]{\hyperref[#1]{#1}} 
\newcommand{\rem}[1]{}
% Marking colors
\definecolor{todo}{rgb}{1,0.2,0.2}
\definecolor{reconsider}{rgb}{0.6,0.6,0.3}
\newcommand{\todo}[1]{{\color{todo} #1}}
\newcommand{\reconsider}[1]{{\color{reconsider} #1}}

% Hurenkinder und Schusterjungen verhindern
\clubpenalty10000
\widowpenalty10000
\displaywidowpenalty=10000


\graphicspath{{images/}{./}}

\title{Aktuelle Anforderungen an Fahrerassistenzsysteme \titlenote{ \scriptsize \flushleft Betreuer Hochschule:  \  \ Prof. Dr. Martinez\\ \qquad \qquad \qquad \qquad \quad \ \  Hochschule Reutlingen\\ \qquad \quad \quad \quad \qquad \qquad \ \ Natividad.Martinez@Reutlingen-\\ \qquad \qquad \qquad \qquad \quad \ \ University.de\\  Informatics Inside 2015 II\\ Wissenschaftliche Vertiefungskonferenz \\ 18. November 2015, Hochschule Reutlingen\\  \copyright 2015 Paul Pasler}}


\numberofauthors{3}
\author{
	\alignauthor
	  \center
		\aufnt{Paul Pasler}\\
          \affaddr{Reutlingen University}\\
        \textbf{\textsf{Paul.Pasler@Student.Reutlingen-University.DE}}
}



\begin{document}

\begin{acronym}
\acro{FAS}{Fahrerassistenzsystem}
\acro{FASs}{Fahrerassistenzsysteme}
\acro{ME}{Müdigkeitserkennung}
\acro{MESs}{Müdigkeitserkennungssysteme}
\acro{ADAS}{Advanced Driver Assistance System}
\acro{ADASs}{Advanced Driver Assistance Systems}
\acro{bspw}{beispielsweise}
\end{acronym}


\maketitle
\sloppypar{
\begin{abstract}
Mit Fortschreiten der Technik, verbreiten sich \acl{FAS}en immer weiter. Auch die \acl{ME} ist ein solches System. In einem kurzen Überblick, wird  die Funktionsweise und verschiedene Umsetzungen von Systemen zur \acl{ME} vorgestellt. Der Ansatz Erkennung mit Body-Sensorik wird auf seine Umsetzbarkeit im Simulationsumfeld der Reutlingen University evaluiert.

\end{abstract}

\keywords{
\acf{ADAS}, \acl{FAS}, \acl{ME}
}

\category{A.0}{ACM}{
Experimentation
}

\section{Einleitung}
\label{chap:intro}
Fristeten \acl{FASs} vor wenigen Jahren ein Nischendasein in Oberklassewagen, werden sie immer günstiger und beliebter. So halten sie auch in Mittel- und Kleinwagen Einzug und helfen bei der Vermeidung schwerer Unfälle. Einen Überblick zur aktuellen Entwicklung bei \acl{FASs} gibt es in Abschnitt \ref{sec:fas}.

Um ein neues System auszuprobieren, setzen Automobilhersteller, wie auch Wissenschaftler, auf Tests in einem Simulator (unter Laborbedingungen), bevor es in echte Fahrzeuge integriert werden. Eine solche Simulationsumgebung, wie sie an der Reutlingen University genutzt wird, wird in Abschnitt \ref{sec:sim} vorgestellt.

In der vorgestellten Arbeit wird die \acl{ME} als Vertreter eines  \acl{FAS}s näher beleuchtet und verschieden Umsetzungsvorschläge gegenübergestellt (Kapitel \ref{chap:me}). In Kapitel \ref{chap:eval} wird ein solches System mit Körpersensoren (EEG, EKG) evaluiert und im Simulationsumfeld der Reutlingen University getestet. Eine Zusammenfassung und weitere Schritte werden in Kapitel \ref{chap:outro} beschrieben.

\subsection{\acl{FASs}}
\label{sec:fas}

- Studie zu Unfällen -
 \cite{Feld:2010:MIA:1719970.1720063}

\acl{FASs} bieten dem Fahrer zum einen ein Komfortplus oder erhöhen die Sicherheit beim Fahren. So führen Einparkassistent,  Geschwindigkeitsregelanlage oder Navigation zu deutlich entspannterem Fahren. Spurhalte-, Spurwechsel- oder Notbremsassistent wiederum unterstützen bei potentiell gefährlichen Manövern. Auch die \acl{ME} fällt in die zweite Kategorie (mehr dazu in Kapitel \ref{chap:me}).\\

\textbf{Überblick und Klassifizierung} \\
\acl{FASs} bieten dem Fahrer zum einen ein Komfortplus oder erhöhen die Sicherheit beim Fahren. So führen Einparkassistent, Tempomat oder Navigation zu deutlich entspannterem Fahren. 
Spurhalte-, Spurwechsel- oder Notbremsassistent wiederum unterstützen bei potentiell gefährlichen Manövern. Auch die \acl{ME} fällt in die zweite Kategorie (mehr dazu in Kapitel \ref{chap:me}).

Kompaß \cite{fasFuture} unterteilt \acl{FASs}, gemessen an der Reaktionszeit, in Planung, Führung und Stabilisierung. Hierbei fällt \acl{bspw} Navigation in die Planungsebene, da die Berechnung der Route mit unter mehrere Minuten brauchen kann. Auf Führungsebene werden dem Fahrer Empfehlungen und Warnungen innerhalb weniger Sekunden mitgeteilt, auf die er dann reagieren kann. Greift das System selbständig in den Fahrprozess ein, muss dies meist innerhalb von Millisekunden geschehen und dient oftmals zur Stabilisierungen, wie \acl{bspw} bei einem Fahrdynamik-Regelsystem.\\

\textbf{Rückmeldung} \\
Ein \acl{FAS} kann auf verschiedenste Arten mit dem Fahrer kommunizieren. Es handelt sich um eine klassische HCI-Schnittstelle. Am gebräuchlichsten, auch für sonstige Warnungen, sind schon seit längerem Optische und Akustische Signale. Aber auch Vibrationen in Lenkrad und Sitz zeigen gute Ergebnisse, wenn zwischen Signal und Nachricht ein Zusammenhang besteht (Bspw. Vibriert das Lenkrad bei verlassen der Spur).
\cite{Bertoldi:2010:MAD:2002368.2002370} beschreibt hierzu die verschiedenen Anwendungsgebiete und Unterschiede. \\

\textbf{Systeme} \\
Jeder Automobilhersteller entwickelt mittlerweile seine eigenen \acl{FASs}. Datenerhebung (Sensoren), Berechnung und Kommunikation werden vom Fahrzeug selbst durchgeführt. Durch die Abschottung des Fahrzeugs sind Fahrzeugdaten nicht öffentlich zugänglich und können nur schwer von Außenstehenden genutzt werden. 

Für wissenschaftliche Arbeiten bleibt entweder eine Kooperation mit Automobilherstellern oder das Ausweichen auf andere Devices, wie ein Smartphone und die Nutzung von Daten aus dem Internet (\acl{bspw} Kartendienste). Chen \cite{Chen:2015:ISV:2742647.2742659} und You \cite{You:2013:CAA:2462456.2465428} verfolgten diesen Ansatz. Smartphone bieten durch ihren hohen Verbreitungsgrad eine günstige Alternative zu eingebauten Systemen, können jedoch nicht auf Daten des Fahrzeugs zugreifen und müssen einfache Daten, wie \acl{bspw} Geschwindigkeit, selbst berechnen.

\subsection{Körpersensoren: EEG, EKG}
\label{sec:sens}
Körpersensoren messen verschiedenste Werte eines lebenden Körpers, wie den Puls, Temperatur oder Hirnwellen. Meistens werden sie direkt am oder im Körper eingesetzt.
Bei der Elektroenzephalografie (EEG) werden Elektroden auf der Kopfhaut angebracht und damit die Aktivität des Gehirns gemessen. Sie wird in der Medizin für die Diagnose von Epilepsie oder bei Komapatienten eingesetzt. Zudem findet sie in Schlaflabors Anwendung, um verschiedene Schlafphasen zu erfassen. Der Zusammenhang von Schlaf und Hirnaktivität kann auch bei der \acl{ME} in Fahrzeugen genutzt werden, um \acl{bspw} ein drohenden Sekundenschlaf zu erkennen.
Das Elektrokardiogramm (EKG) misst die Herzspannungskurve und stellt die Aktivität des Herzmuskels dar. So lassen sich vielfältige Aussagen über den Zustand des Herzens machen. Weiterhin können Herzrythmus und -frequenz Hinweise auf eine einfallende Müdigkeit des Fahrers geben. 


\subsection{Simulationsumgebung}
\label{sec:sim}
\begin{itemize}
  \item Aufbau des Simulators
  \item Technische Kommunikation im Fahrzeug / Simulator \cite{serial}
\end{itemize}



\section{Systeme zur \acl{ME}}
\label{chap:me}
Müdigkeit senkt die Konzentrationsfähigkeit des Fahrers und kann zu einer erhöhten Reaktionszeit und Fehleinschätzungen führen, wie es der Deutscher Verkehrssicherheitsrat in einem Beschluss von 2009 \cite{DVR:Online} feststellt.
Die \acl{ME} versucht an Hand verschiedener Daten und Sensoren, frühzeitig zu erkennen, ob der Fahrer gerade Anzeichen einer bevorstehenden Müdigkeit zeigt und empfiehlt eine Pause.
Dabei soll nicht nur während Micro- oder Sekundenschlaf, sondern schon früher gewarnt werden. \\

\textbf{Überblick und Klassifizierung} \\
Die Erkennung von Müdigkeit kann auf ganz verschiedene Arten gelöst werden. Ein Ansatz versucht über Körpersignale herauszufinden, ob eine Müdigkeit bevorsteht. Wohingegen mit der Analyse des Fahrverhaltens das Selbe mit Sensoren an und im Auto realisiert wird.
Bei der Erkennung über Körpersignale, können wiederum Körpersensoren oder  Computer-Vision-Techniken zur Überwachung des Fahrers genutzt werden. Zu unterscheiden ist weiterhin die physische und psychische Müdigkeit, welche sich jedoch beide negativ auf die Fähigkeiten des Fahrers auswirken. Alle Verfahren, die auf Senoren am Körper, die extra angezogen werden (bspw. ein Pulsmesser am Ohr, EEG) werden als inversive Verfahren bezeichnet.

Allen System gemein ist die Nutzung von Klassifizierungs- bzw. Machine-Learning-Algorithmen. Die gesammelten Daten geben nur Hinweise und sind kein Garant für eine Erkennung von Müdigkeit. \acl{MESs} wandeln hier auf einem schmalen Grad, da es zum einen um die Verhinderung schwerer Unfälle geht, zum anderen aber ein falsch auslösendes System die Akzeptanz vermindert und im schlimmsten Fall zu einer Deaktivierung führt. Um falsche Erkennungen weiter zu minimieren, werden oftmals mehrere Ansätze kombiniert.

In der Praxis setzen Automobilhersteller wie Daimler \cite{Daimler} und Volkswagen, sowie Automobilzulieferer wie Bosch \cite{Bosch} auf die Analyse des Fahrverhaltens. Insbesondere Spurhalten und ruckartiges Gegenlenken scheinen ein signifikantes Indiz für beginnende Übermüdung zu sein. Weiterhin sind externe Geräte und einige Apps für Smartphones erhältlich. \\

\textbf{Stand der Technik} \\
Wie bereits erwähnt, existieren verschiedene Vorgehen, für die \acl{ME} im Fahrzeug. Ansätze mit Körpersensoren liefern gute Ergebnisse, müssen jedoch am Körper angebracht werden, was zu einer Beeinträchtigung führen kann.\\

Bundele and Banerjee \citep{Bundele:2009:DFV:1806338.1806478} zeigten, dass Müdigkeit über die elektrodermale Hautreaktion und Pulsoxymetrie erkannt werden kann. Die elektrodermale Hautreaktion (galvanische Hautreaktion, GSR) misst hierbei die Hautleitfähigkeit und hängt mit der Schweißproduktion zusammen. Bei der Pulsoxymetrie kann, durch ein optische Verfahren, die Sauerstoffsättigung des Blutes gemessen werden. In diesem Fall bedeutet eine geringere Sättigung ein erhöhtes Müdigkeitsgefühl. Diese Werte werden durch Körpersensoren ermittelt und werden mit einem Multi Layer Perceptron (MLP) klassifiziert. Interessant ist zudem der Einsatz von sogenannten Smart-Clothes (E-textiles), welche die Sensoren in der Kleidung eingearbeitet haben und somit zu ein Non-Inversiven Ansatz führen.\\

TODO\\
Johnson et. al \cite{Johnson11} nutzen EEG.
Ronzhina et al. \cite{Ronzhina:2011:UEV:2093698.2093733} evaluierten den Flicker-Fusion Test \\

Park et al. \cite{Park:2009:DDD:1667780.1667798} beschränken sich in ihrer Arbeit auf die Analyse der Pulswelle durch Photoplethysmography (PPG), mit einem eingebauten Sensor am Lenkrad. Dies stellt schon ein größeren Eingriff in die Umgebung des Fahrzeugs dar, als es im Abschnitt zuvor der Fall war. Die Daten der PPG werden mit einer Support Vector Maschine eingeordnet. Es zeigte sich, dass die Ausschlagshöhe des Puls in gute Mittel für die Erkennung von Müdigkeit darstellt. Um die Ergebnisse zu verbessern, wurde die entwickelte Software mit einem zuvor entwickelten visuellen System zur Kopfbewegung gekoppelt. \\

Neben den Ansätzen mit Körpersensoren, existieren weitere Ansätze die mit Hilfe von Kameras den Fahrer und die Straße beobachten. Zhang et al. \cite{Zhang:2015:RSD:2753829.2629482} stellen hierzu eine Applikation mit der Verbindung eines Farb- und Tiefenbildes vor. Mit Hilfe einer Microsoft Kinect werden sowohl die Kopfpose, als auch die Augenstatus bestimmt. Um das System robuster zu gestalten, wird aus dem Farbbild,  zusätzlich zum Vorhandenen, das Tiefenbild berechnet.\\

Mit der CarSafe App entwickleten You et al. \cite{You:2013:CAA:2462456.2465428} ein visuelles System zur Überwachung des Fahrers und der Straße. Hierfür genügt ein aktuelles Smartphone. Die App deckt hierbei neben der \acl{ME} auch andere Gefahrensituation (\acl{bspw} zu dicht Auffahren) ab. Es werden eine Analyse des Fahrers (Kopfpose und Augenstatus), sowie der Fahrweise kombiniert und der Fahrer gewarnt. Kamerabasierte Systeme sind angenehm für den Fahrer, da er keine weitere Hardware (Sensoren) installieren muss. Jedoch ist eine Kamera optischen Grenzen unterworfen, was den Einsatz bei Nacht oder schlechtem Wetter erschwert. Für eine \acl{ME} mit Smartphone, aber ohne Kameraeinsatz, könnte \acl{bspw} die App V-Sense \cite{Chen:2015:ISV:2742647.2742659} genutzt werden, da sie lediglich  eingebaute Sensoren nutzt.

\section{Evaluation einer \acl{ME} im Simulationsumfeld}
\label{chap:eval}

\begin{itemize}
  \item Unterschiede / Einschränkungen echtes Fahrzeug / Simulator
  \item Versuchsaufbau
  \item Ergebnis / Evtl. Prototyp
\end{itemize}

\section{Fazit}
\label{chap:outro}
\begin{itemize}
  \item Weitere Schritte
\end{itemize}


\balance
\bibliographystyle{abbrv} % abbrv, alpha, plain, unsrt, apalike
\bibliography{Quellen,Zotero}


\end{document}
